\documentclass[10pt]{amsart}
\usepackage{graphicx} 
\usepackage{amsmath}
\usepackage{pdflscape}
\usepackage{tabularx}
\usepackage{endnotes}
\usepackage{gensymb}
\usepackage{verbatimbox}
\usepackage{float} % necessary for placement of figures
\usepackage[style = authoryear, sorting = nyt, backend = biber]{biblatex}
%\addbibresource[location = local, type = file]{C:/Users/bloh356/Google Drive/Library/Library.bib}
\addbibresource[location = local, type = file]{/Users/bloh356/Google Drive/Library/Library.bib}

\title{$O_{3}$ task}
\author{Andrew Blohm}
\date{\today}

\begin{document}
\maketitle

\section{Abstract}

\section{Literature review}
Ozone ($O_{3}$) negatively effects plant growth and physiology, and subsequently, yields, as it causes injury, reduces photosynthesis, and, decreases plant productivity \parencite{mishra:2013aa}. 

There are two main approaches used to quantify the impact of ozone on crops: concentration based critical levels ($CLe_{c}$) (i.e., AOT40), which capture the relationship between the cumulative dose over a threshold on the outcome variable, and flux models, which model the flux of ozone through the stomatal pores to identify critical levels of ozone \parencite{mills:2007aa}. 
A third, less used model, is the maximum permissible ozone concentration (MPOC)\parencite{mills:2007aa}.  
Evidence in Europe seems to support the idea that flux-based assessments do a better job than AOT40 exposure in predicting ozone impacts \parencite{avnery:2013ab}.\footnote{As of 2013, the flux-based models were not suitable for global impact analysis \parencite{avnery:2013ab}. The authors applied the AOT40 and W126 relationships developed for the US and Europe globally.} 

The critical load and level concept, an effect-based approach to quantify the necessary level of reductions of atmospheric depositions, came about as a result of the Convention on Long-range Transboundary Air Pollution (LRTAP-Convention) \parencite{clrtap:2015aa}. 
Critical levels are defined as the "ambient concentrations above which damage may occur" \parencite[I-5]{clrtap:2015aa}.
"Critical loads and levels correspond to a maximum allowable exposure of a receptor to deposition or ambient concentration respectively" \parencite[I-5]{clrtap:2015aa}.
Note: We could add more information here on the convention, organizations, monitoring, etc. 

The $CLe_{c}$ approach uses the accumulated ozone above a threshold concentration for a certain time-window. 
As a first step these studies identify the critical level of accumulated ozone above which, damage occurs \parencite{mills:2007aa}.
One criticism of this type of model is that it does not take into account how much of the ozone actually enters the plant through the stomatal pores.
Flux modeling addresses this issue derives the flux-based critical levels ($CLe_{f}$) \parencite{mills:2007aa}.  

Four major studies have been undertaken that serve as the basepoint for studies of this type (and their resulting estimates): US National Crop Loss Assessment Network (1980s); the European Open Top Chamber Programme (EOTCP) (late 1980s to early 1990s), open-top chamber (OTC) experiments in Europe, and the Changing Climate and Potential Impact on Potato Yield and Quality (year?) \parencite{mills:2007aa}. 

From the NCLAN and EOTCP data researchers identified a cumulative exposure threshold of 50-87 ppb in the United States and 35-60 ppb in Europe \parencite{mills:2007aa}.
This work led to the use of the AOT40 measure, which is the lowest levels at which crop damage could be detected, in Europe, as the concentration based indicator of ozone damage to crops.

AOT40 refers to the concentration of ozone accumulated over a threshold concentration of ozone, which us usually 30 or 40 ppb (i.e., AOT30 and AOT40, respectively). 
AOTX is calculated as the sum of the differences between the hourly mean ozone level and \textit{'X'} for all daylight hours over a time-window, in which the hourly mean is greater than \textit{'X'} \parencite{mills:2007aa}.
The units of AOTX are parts per billion per hour.\footnote{The mapping manual 'LRTAP Convention (2004)' has detailed description of how to calculate the AOTX.}
One criticism of the concentration threshold approach is that it only considers the ozone levels at the top of the canopy, which directly led to work on an alternative measure (i.e., flux-based critical levels). 

\cite{mills:2007aa} through a metaanlysis of over 700 published papers and proceedings found that crops can be separated into three statistically independent groups: ozone sensitive (i.e., wheat, water melon, pulses, cotton, turnip, tomato, onion, soybean, and lettuce), moderately sensitive (sugar beet, potato, oilseed rape, tobacoo, rice, maize, grape, and broccoli), and ozone resistant (i.e., barley and fruit (e.g. plum and strawberry)) \parencite{mills:2007aa}.  
Wheat is the most sensitive crop to $O_{3}$ levels \parencite{mills:2007aa}. 

There is a small literature attempting to quantify the impacts of ozone on agricultural production. 
Changes in ozone can be estimated using either a chemical transport model or using source-receptor relationships established through an adjoint model. 
The chemical transport model can be computationally expensive given the need to run the model for each representative concentration pathway scenario.
Whereby the adjoint model can estimate the necessary variables for multiple scenarios during one model run \parencite{lapina:2015aa}.
The adjoint model establishes the sensitivities (i.e., a fractional response to a pertubation in the source at a particular location) of a particular receptor metric to parameters of the chemical transport model (i.e., ozone, emissions, etc.).
The authors assume that the response is time invariant, which means that to determine a future sensitivity per fractional change in emissions only requires an understanding of how the emission levels change \parencite{lapina:2015aa}.
The authors used the Hemispheric Transport of Air Pollution (HTAP) study to estimate the W126 response to changing levels of methane.

\cite{tai:2014aa} use the Community Earth System Model (CESM) to simulate hourly temperature and ozone concentrations consistent with representative concentration pathway (RCP) scenarios 4.5 and 8.5 for present day and 2050.\footnote{Employed a coupled atmosphere and land components with fixed data ocean and cryosphere consistent with current and future climates at a resolution of 1.9\degree by 2.5\degree.}
The authors model both the effect of future climate on agriculture through agro-climate variables (i.e., growing and killing degree days), as well as the detrimental effect of ozone.\footnote{The authors used four ozone exposure indices: AOT40, SUM06, W126, and M7 or M12. A constrained linear regression is used to determine the regional relationship between these variables and historical observations while also controlling for the correlation between high temperatures and ozone levels. The projections exclude the potential effect of $CO_2$ on global crop production.}  

\cite{lapina:2015aa} use GEOS-Chem adjoint to calculate regional W126 concentrations using the RCP emissions pathways and source-receptor relationships for multiple RCP scenarios as calculated through the GEOS-Chem adjoint model \footnote{The source-receptor relationship, as estimated through adjoint models, is less computationally intensive as compared to chemical transport models because the relationship can be established for multiple emission sources during a single adjoint model run \parencite{lapina:2015aa}. The model uses the GEOS-5 assimilated meteorology with 2\degree by 2.5\degree resolution and 47 vertical levels}.
The authors also consider the impacts of changes in methane emissions on ozone levels.

\cite{avnery:2013ab} investigate two mitigation alternatives to reduce the expected agricultural losses to ozone level: methane emission reduction and reducing crop damage through the use of more ozone resilient varieties.
The authors use the Mozart-2 global chemical transport model to simulate and compare changes in surface ozone and resulting crop loss, as a result of future methane emissions to several mitigation alternatives under current legislation (CLE) and reduced methane scenarios (i.e., methane reduction and choosing crop varieties resistant to ozone). 
NCEP reanalysis is used for the meteorological fields to drive the simulations with a resolution of 1.9\degree by 1.9\degree horizontal resolution and 28 vertical levels \parencite{avnery:2013ab}.
The authors use the W126 and AOT40 measures.

\cite{wang:2004aa} use MOZART-2 to simulate ozone levels for the period 1990 to 2020. 
The model is driven by meteorological inputs from the GCM Middle Atmosphere Community Climate Model Version 3 (MACCM3) and has a horizontal resolution of 2.8\degree by 2.8\degree and includes 24 vertical levels \parencite{wang:2004aa}. 
Emissions projections are derived from the historical record and the IPCC B2-Message scenario. 


\begin{landscape}
\begin{table}
\resizebox{\linewidth}{!}{%
	\begin{tabularx}{1.2\linewidth}{l l l l l l l l}
	\hline
	Crop & Yield based on weight of & Critical level (ppm h, 3 months) & No. of cultivars & No. of points & Function & $r^2$ & References used \\
	\hline
	Watermelon & Fruit & 1.6 & 1 & 4 & y = -0.0321x + 0.97 & 0.94 & Gemino et al. (1999) \\
	\end{tabularx}}
\end{table}
\end{landscape}

\section{Data}
Ozone monitoring networks are in place at many locations around the world; however, much of the coverage is in North America and Europe. 
In 2016 \cite{sofen:2016aa} compiled, processed, and made publicly available an ozone dataset that spans 2200 high quality sites with over 200 million hourly observations for the period 1971-2015.
The project released the data as gridded metrics on the monthly and annual timescales.
Metrics available include the maximum daily 8-hour average, sum of means over 35 ppb, accumulated ozone exposure above a threshold of 40 ppb, etc. 
More information on the complete list of metrics available can be found in \cite{sofen:2016aa}.

For our analysis we use the 1\degree x 1\degree gridded AOT40 crop product, which is calculated as the accumulated ozone over a threshold of 40 ppb during the May to July growing season between the hours of 8:00 am and 2:00 pm \parencite{sofen:2016aa}.

\cite{} investigate health impacts of ozone using simulations of future climate. 
The \cite{j:2013aa} paper generated an annual average for PM2.5 and a 6-month (I assume growing season based) average of the 1-hour daily maximum ozone. 
If I understand correctly these values were then averaged over four years. 
The authors use the MOZART-4 global chemical transport model to simulate the level of ozone and PM2.5 in 2000, 2030, 2050, and 2100 using the Global GHG emission reduction in the Representative Concentration Pathway 4.5 (RCP4.5) scenario, as well as the associated reference scenario for the pathway.
The authors use the AM3 model to incorporate the meterorological inputs from the global general circulation model simulations of RCP4.5.  



The data was intended to be at the 2 degree x 2.5 degree grid but the calculations were messed up so we'd need to ask them what the grid size actually is. 


\section{Methods}


\printbibliography
\end{document}